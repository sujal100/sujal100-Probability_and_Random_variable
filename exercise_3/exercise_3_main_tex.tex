\documentclass[journal,12pt,twocolumn]{IEEEtran}

\usepackage{setspace}
\usepackage{gensymb}
\singlespacing
\usepackage[cmex10]{amsmath}
\usepackage{multirow}
\usepackage{amsthm}
\usepackage{mathrsfs}
\usepackage{txfonts}
\usepackage{stfloats}
\usepackage{bm}
\usepackage{cite}
\usepackage{cases}
\usepackage{subfig}

\usepackage{longtable}
\usepackage{amsmath}
\usepackage{enumitem}
\usepackage{mathtools}
\usepackage{steinmetz}
\usepackage{tikz}
\usepackage{circuitikz}
\usepackage{verbatim}
\usepackage{tfrupee}
\usepackage[breaklinks=true]{hyperref}
\usepackage{graphicx}
\usepackage{tkz-euclide}

\usetikzlibrary{calc,math}
\usepackage{listings}
    \usepackage{color}                                            %%
    \usepackage{array}                                            %%
    \usepackage{longtable}                                        %%
    \usepackage{calc}                                             %%
    \usepackage{multirow}                                         %%
    \usepackage{hhline}                                           %%
    \usepackage{ifthen}                                           %%
    \usepackage{lscape}     
\usepackage{multicol}
\usepackage{chngcntr}

\DeclareMathOperator*{\Res}{Res}

\renewcommand\thesection{\arabic{section}}
\renewcommand\thesubsection{\thesection.\arabic{subsection}}
\renewcommand\thesubsubsection{\thesubsection.\arabic{subsubsection}}

\renewcommand\thesectiondis{\arabic{section}}
\renewcommand\thesubsectiondis{\thesectiondis.\arabic{subsection}}
\renewcommand\thesubsubsectiondis{\thesubsectiondis.\arabic{subsubsection}}


\hyphenation{op-tical net-works semi-conduc-tor}
\def\inputGnumericTable{}                                 %%

\lstset{
%language=C,
frame=single, 
breaklines=true,
columns=fullflexible
}
\begin{document}


\newtheorem{theorem}{Theorem}[section]
\newtheorem{problem}{Problem}
\newtheorem{proposition}{Proposition}[section]
\newtheorem{lemma}{Lemma}[section]
\newtheorem{corollary}[theorem]{Corollary}
\newtheorem{example}{Example}[section]
\newtheorem{definition}[problem]{Definition}

\newcommand{\BEQA}{\begin{eqnarray}}
\newcommand{\EEQA}{\end{eqnarray}}
\newcommand{\define}{\stackrel{\triangle}{=}}
\bibliographystyle{IEEEtran}
\raggedbottom
\setlength{\parindent}{0pt}
\providecommand{\mbf}{\mathbf}
\providecommand{\pr}[1]{\ensuremath{\Pr\left(#1\right)}}
\providecommand{\qfunc}[1]{\ensuremath{Q\left(#1\right)}}
\providecommand{\sbrak}[1]{\ensuremath{{}\left[#1\right]}}
\providecommand{\lsbrak}[1]{\ensuremath{{}\left[#1\right.}}
\providecommand{\rsbrak}[1]{\ensuremath{{}\left.#1\right]}}
\providecommand{\brak}[1]{\ensuremath{\left(#1\right)}}
\providecommand{\lbrak}[1]{\ensuremath{\left(#1\right.}}
\providecommand{\rbrak}[1]{\ensuremath{\left.#1\right)}}
\providecommand{\cbrak}[1]{\ensuremath{\left\{#1\right\}}}
\providecommand{\lcbrak}[1]{\ensuremath{\left\{#1\right.}}
\providecommand{\rcbrak}[1]{\ensuremath{\left.#1\right\}}}
\theoremstyle{remark}
\newtheorem{rem}{Remark}
\newcommand{\sgn}{\mathop{\mathrm{sgn}}}
\providecommand{\abs}[1]{\left\vert#1\right\vert}
\providecommand{\res}[1]{\Res\displaylimits_{#1}} 
\providecommand{\norm}[1]{\left\lVert#1\right\rVert}
%\providecommand{\norm}[1]{\lVert#1\rVert}
\providecommand{\mtx}[1]{\mathbf{#1}}
\providecommand{\mean}[1]{E\left[ #1 \right]}
\providecommand{\fourier}{\overset{\mathcal{F}}{ \rightleftharpoons}}
%\providecommand{\hilbert}{\overset{\mathcal{H}}{ \rightleftharpoons}}
\providecommand{\system}{\overset{\mathcal{H}}{ \longleftrightarrow}}
	%\newcommand{\solution}[2]{\textbf{Solution:}{#1}}
\newcommand{\solution}{\noindent \textbf{Solution: }}
\newcommand{\cosec}{\,\text{cosec}\,}
\providecommand{\dec}[2]{\ensuremath{\overset{#1}{\underset{#2}{\gtrless}}}}
\newcommand{\myvec}[1]{\ensuremath{\begin{pmatrix}#1\end{pmatrix}}}
\newcommand{\mydet}[1]{\ensuremath{\begin{vmatrix}#1\end{vmatrix}}}
\newcommand*{\permcomb}[4][0mu]{{{}^{#3}\mkern#1#2_{#4}}}
\newcommand*{\perm}[1][-3mu]{\permcomb[#1]{P}}
\newcommand*{\comb}[1][-1mu]{\permcomb[#1]{C}}
\numberwithin{equation}{subsection}
\makeatletter
\@addtoreset{figure}{problem}
\makeatother
\let\StandardTheFigure\thefigure
\let\vec\mathbf
\renewcommand{\thefigure}{\theproblem}
\def\putbox#1#2#3{\makebox[0in][l]{\makebox[#1][l]{}\raisebox{\baselineskip}[0in][0in]{\raisebox{#2}[0in][0in]{#3}}}}
     \def\rightbox#1{\makebox[0in][r]{#1}}
     \def\centbox#1{\makebox[0in]{#1}}
     \def\topbox#1{\raisebox{-\baselineskip}[0in][0in]{#1}}
     \def\midbox#1{\raisebox{-0.5\baselineskip}[0in][0in]{#1}}
\vspace{3cm}
\title{Assignment 3}
\author{Sujal - AI20BTECH11020}
\maketitle
\newpage
\bigskip
\renewcommand{\thefigure}{\theenumi}
\renewcommand{\thetable}{\theenumi}
Download all python codes from 
\begin{lstlisting}
https://github.com/sujal100/Probability_and_Random_variable/tree/main/exercise_3/codes
\end{lstlisting}

and latex codes from 

\begin{lstlisting}
https://github.com/https://github.com/sujal100/Probability_and_Random_variable/blob/main/exercise_3/exercise_3_main_tex.tex
\end{lstlisting}
\section{Problem [GATE(2015)MA-11]}
In an experiment, a fair die is rolled until two sixes are obtained in succession. The probability that the experiment will end in the fifth trial is equal to\\
(A) $\frac{125}{6^{5}}$
(B) $\frac{150}{6^{5}}$
(C) $\frac{175}{6^{5}}$
(D) $\frac{200}{6^{5}}$
\section{Solution}
Let Consider, Bernoulli random variables say $X$. Here, $P(X=n)$ refer to the probability that experiment ends in exactly $n^{th}$ rolls. Thus, problem is asking for $p_5.$ We note that
\begin{align}
p_1=0, p_2=\frac{1}{6^{2}}
\end{align} 
\begin{figure}
\centering
\tikzset{every picture/.style={line width=0.5pt}} %set default line width to 0.75pt        

\begin{tikzpicture}[x=0.75pt,y=0.75pt,yscale=-0.8,xscale=1]
%uncomment if require:  %set diagram left start at 0, and has height of 300
\path (100,0);
%Straight Lines [id:da22181978815813874] 
\draw    (296,41) -- (404.07,103.2) ;
\draw [shift={(405.8,104.2)}, rotate = 209.92000000000002] [color={rgb, 255:red, 0; green, 0; blue, 0 }  ][line width=0.75]    (10.93,-3.29) .. controls (6.95,-1.4) and (3.31,-0.3) .. (0,0) .. controls (3.31,0.3) and (6.95,1.4) .. (10.93,3.29)   ;
%Straight Lines [id:da161455286927225] 
\draw    (296,41) -- (196.48,105.12) ;
\draw [shift={(194.8,106.2)}, rotate = 327.21000000000004] [color={rgb, 255:red, 0; green, 0; blue, 0 }  ][line width=0.75]    (10.93,-3.29) .. controls (6.95,-1.4) and (3.31,-0.3) .. (0,0) .. controls (3.31,0.3) and (6.95,1.4) .. (10.93,3.29)   ;
%Straight Lines [id:da2511419168154414] 
\draw    (123.8,133.2) -- (151.8,133.2) ;
%Straight Lines [id:da06439769803839601] 
\draw    (165.8,133.2) -- (193.8,133.2) ;
%Straight Lines [id:da14609392755284478] 
\draw    (341.8,134.2) -- (369.8,134.2) ;
%Straight Lines [id:da38196067412588475] 
\draw    (136.8,139.2) -- (137.7,158.2) ;
\draw [shift={(137.8,160.2)}, rotate = 267.27] [color={rgb, 255:red, 0; green, 0; blue, 0 }  ][line width=0.75]    (10.93,-3.29) .. controls (6.95,-1.4) and (3.31,-0.3) .. (0,0) .. controls (3.31,0.3) and (6.95,1.4) .. (10.93,3.29)   ;
%Straight Lines [id:da6427254369866533] 
\draw    (175.8,139.2) -- (176.7,158.2) ;
\draw [shift={(176.8,160.2)}, rotate = 267.27] [color={rgb, 255:red, 0; green, 0; blue, 0 }  ][line width=0.75]    (10.93,-3.29) .. controls (6.95,-1.4) and (3.31,-0.3) .. (0,0) .. controls (3.31,0.3) and (6.95,1.4) .. (10.93,3.29)   ;
%Straight Lines [id:da7865775583247143] 
\draw    (356.8,139.2) -- (357.7,158.2) ;
\draw [shift={(357.8,160.2)}, rotate = 267.27] [color={rgb, 255:red, 0; green, 0; blue, 0 }  ][line width=0.75]    (10.93,-3.29) .. controls (6.95,-1.4) and (3.31,-0.3) .. (0,0) .. controls (3.31,0.3) and (6.95,1.4) .. (10.93,3.29)   ;

% Text Node
\draw (255,15) node [anchor=north west][inner sep=0.75pt]   [align=left] {For};
% Text Node
\draw (124,114.4) node [anchor=north west][inner sep=0.75pt]    {$1/6$};
% Text Node
\draw (98,389.4) node [anchor=north west][inner sep=0.75pt]    {$ $};
% Text Node
\draw (283,15.4) node [anchor=north west][inner sep=0.75pt]    {$pr_x(n)$};
% Text Node
\draw (164,114.4) node [anchor=north west][inner sep=0.75pt]    {$5/6$};
% Text Node
\draw (341,115.4) node [anchor=north west][inner sep=0.75pt]    {$5/6$};
% Text Node
\draw (203,112.4) node [anchor=north west][inner sep=0.75pt]    {$\underbrace{p_x(n-2)}_{n-2}$};
% Text Node
\draw (385,113.4) node [anchor=north west][inner sep=0.75pt]    {$\underbrace{p_x(n-1)}_{n-1}$};
% Text Node
\draw (307,116) node [anchor=north west][inner sep=0.75pt]   [align=left] {OR};
% Text Node
\draw (104,157) node [anchor=north west][inner sep=0.75pt]   [align=left] {Here 6 \\ occur};
% Text Node
\draw (163,157) node [anchor=north west][inner sep=0.75pt]   [align=left] {Here 6\\ \ not\\ occur};
% Text Node
\draw (337,158) node [anchor=north west][inner sep=0.75pt]   [align=left] {Here 6\\ \ not\\ occur};
% Text Node
\draw (149,114.4) node [anchor=north west][inner sep=0.75pt]    {$\times $};
% Text Node
\draw (188.8,113.6) node [anchor=north west][inner sep=0.75pt]    {$\times $};
% Text Node
\draw (368,114.4) node [anchor=north west][inner sep=0.75pt]    {$\times $};
\end{tikzpicture}
\label{fig:my_label}
\end{figure}

For $n>2,$ we remark that the first roll is either a 6 or it isn't. If it is, then the second roll can't be a 6. That leads to the recursion
\begin{align}
p_n=\frac{1}{6} \times \frac{5}{6} \times p_{n-2}+\frac{5}{6} \times p_{n-1}
\end{align}
Put $n=0,$ we get
\begin{align}
&p_2=\frac{5}{6^{2}}p_0+\frac{5}{6}p_1\\
&\frac{1}{6^{2}}=\frac{5}{6^{2}}p_0\\
&p_0=\frac{1}{5}
\end{align}
Now, put n=n+2 
We have,
\begin{align}
p_{n+2}=\frac{5}{6^{2}}p_{n}+\frac{5}{6}p_{n+1},p_0=\frac{1}{5},p_1=0
\end{align}
Taking, Z-transform both side,
\begin{align}
Z[p_{n+2}]=&\frac{5}{6^{2}}Z[p_{n}]+\frac{5}{6}Z[p_{n+1}]\\
z^2P(z)-z^2p_0-zp_1=&\frac{5}{6^2}P(z)+\frac{5}{6}[zP(z)-zp_0]
\end{align}
Inserting initial conditions,
\begin{align}
z^2P(z)-\frac{z^2}{5}=&\frac{5}{6^2}P(z)+\frac{5z}{6}P(z)-\frac{z}{6} \\
P(z)=&\dfrac{\brak{\dfrac{z^2}{5}-\dfrac{z}{6^2}}}{\brak{z^2-\dfrac{5z}{6}-\dfrac{5}{6^2}}}\\
\frac{5P(z)}{z}=&\frac{36z-5}{36z^2-5z-5}\\
\frac{5\sqrt{745}P(z)}{36}=&\frac{(5+\sqrt{745})z}{(72 z+\sqrt{745}-5)}-\frac{(\sqrt{745}-5)z}{(-72 z+\sqrt{745}+5)}
\end{align}
We know that $Z^{-1}\brak{\dfrac{z}{z-a}}=a^n$\\
So, taking Z-inverse transform both side,we get
\begin{align}
\frac{5\sqrt{745}}{36}Z^{-1}\brak{P(z)}=\dfrac{(5+\sqrt{745})}{72}Z^{-1}\brak{\frac{z}{z-\frac{(5-\sqrt{745})}{72}}}\\
-\frac{(5-\sqrt{745})}{72}Z^{-1}\brak{\frac{z}{z-\frac{(\sqrt{745}+5)}{72}}}
\end{align}
\begin{align}
\frac{5\sqrt{745}}{36}p_n=\frac{(5+\sqrt{745})}{72}\brak{\frac{(5-\sqrt{745})}{72}}^{n}\\
-\frac{(5-\sqrt{745})}{72}\brak{\frac{(\sqrt{745}+5)}{72}}^{n}
\end{align}
\begin{align}
p_n=\brak{\frac{1}{2\sqrt{745}}+\frac{1}{10}}\brak{\frac{(5-\sqrt{745})}{72}}^{n}\\
-\brak{\frac{1}{2\sqrt{745}}-\frac{1}{10}}\brak{\frac{(\sqrt{745}+5)}{72}}^{n}
\end{align}
so,
\begin{align}
p_5=\brak{\frac{1}{2\sqrt{745}}+\frac{1}{10}}\brak{\frac{(5-\sqrt{745})}{72}}^{5}\\
-\brak{\frac{1}{2\sqrt{745}}-\frac{1}{10}}\brak{\frac{(\sqrt{745}+5)}{72}}^{5}
\end{align}
\begin{align}
p_5=0.001146095297972871
\end{align}
Hence $(\mathrm{C})$ is correct option.
\end{document}