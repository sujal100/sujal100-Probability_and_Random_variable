\documentclass[journal,12pt,twocolumn]{IEEEtran}

\usepackage{setspace}
\usepackage{gensymb}
\singlespacing
\usepackage[cmex10]{amsmath}
\usepackage{multirow}
\usepackage{amsthm}
\usepackage{mathrsfs}
\usepackage{txfonts}
\usepackage{stfloats}
\usepackage{bm}
\usepackage{cite}
\usepackage{cases}
\usepackage{subfig}

\usepackage{longtable}

\usepackage{enumitem}
\usepackage{mathtools}
\usepackage{steinmetz}
\usepackage{tikz}
\usepackage{circuitikz}
\usepackage{verbatim}
\usepackage{tfrupee}
\usepackage[breaklinks=true]{hyperref}
\usepackage{graphicx}
\usepackage{tkz-euclide}

\usetikzlibrary{calc,math}
\usepackage{listings}
    \usepackage{color}                                            %%
    \usepackage{array}                                            %%
    \usepackage{longtable}                                        %%
    \usepackage{calc}                                             %%
    \usepackage{multirow}                                         %%
    \usepackage{hhline}                                           %%
    \usepackage{ifthen}                                           %%
    \usepackage{lscape}     
\usepackage{multicol}
\usepackage{chngcntr}

\DeclareMathOperator*{\Res}{Res}

\renewcommand\thesection{\arabic{section}}
\renewcommand\thesubsection{\thesection.\arabic{subsection}}
\renewcommand\thesubsubsection{\thesubsection.\arabic{subsubsection}}

\renewcommand\thesectiondis{\arabic{section}}
\renewcommand\thesubsectiondis{\thesectiondis.\arabic{subsection}}
\renewcommand\thesubsubsectiondis{\thesubsectiondis.\arabic{subsubsection}}


\hyphenation{op-tical net-works semi-conduc-tor}
\def\inputGnumericTable{}                                 %%

\lstset{
%language=C,
frame=single, 
breaklines=true,
columns=fullflexible
}
\graphicspath{{./Figures/}}
\begin{document}


\newtheorem{theorem}{Theorem}[section]
\newtheorem{problem}{Problem}
\newtheorem{proposition}{Proposition}[section]
\newtheorem{lemma}{Lemma}[section]
\newtheorem{corollary}[theorem]{Corollary}
\newtheorem{example}{Example}[section]
\newtheorem{definition}[problem]{Definition}

\newcommand{\BEQA}{\begin{eqnarray}}
\newcommand{\EEQA}{\end{eqnarray}}
\newcommand{\define}{\stackrel{\triangle}{=}}
\bibliographystyle{IEEEtran}
\raggedbottom
\setlength{\parindent}{0pt}
\providecommand{\mbf}{\mathbf}
\providecommand{\pr}[1]{\ensuremath{\Pr\left(#1\right)}}
\providecommand{\qfunc}[1]{\ensuremath{Q\left(#1\right)}}
\providecommand{\sbrak}[1]{\ensuremath{{}\left[#1\right]}}
\providecommand{\lsbrak}[1]{\ensuremath{{}\left[#1\right.}}
\providecommand{\rsbrak}[1]{\ensuremath{{}\left.#1\right]}}
\providecommand{\brak}[1]{\ensuremath{\left(#1\right)}}
\providecommand{\lbrak}[1]{\ensuremath{\left(#1\right.}}
\providecommand{\rbrak}[1]{\ensuremath{\left.#1\right)}}
\providecommand{\cbrak}[1]{\ensuremath{\left\{#1\right\}}}
\providecommand{\lcbrak}[1]{\ensuremath{\left\{#1\right.}}
\providecommand{\rcbrak}[1]{\ensuremath{\left.#1\right\}}}
\theoremstyle{remark}
\newtheorem{rem}{Remark}
\newcommand{\sgn}{\mathop{\mathrm{sgn}}}
\providecommand{\abs}[1]{\left\vert#1\right\vert}
\providecommand{\res}[1]{\Res\displaylimits_{#1}} 
\providecommand{\norm}[1]{\left\lVert#1\right\rVert}
%\providecommand{\norm}[1]{\lVert#1\rVert}
\providecommand{\mtx}[1]{\mathbf{#1}}
\providecommand{\mean}[1]{E\left[ #1 \right]}
\providecommand{\fourier}{\overset{\mathcal{F}}{ \rightleftharpoons}}
%\providecommand{\hilbert}{\overset{\mathcal{H}}{ \rightleftharpoons}}
\providecommand{\system}{\overset{\mathcal{H}}{ \longleftrightarrow}}
	%\newcommand{\solution}[2]{\textbf{Solution:}{#1}}
\newcommand{\solution}{\noindent \textbf{Solution: }}
\newcommand{\cosec}{\,\text{cosec}\,}
\providecommand{\dec}[2]{\ensuremath{\overset{#1}{\underset{#2}{\gtrless}}}}
\newcommand{\myvec}[1]{\ensuremath{\begin{pmatrix}#1\end{pmatrix}}}
\newcommand{\mydet}[1]{\ensuremath{\begin{vmatrix}#1\end{vmatrix}}}
\newcommand*{\permcomb}[4][0mu]{{{}^{#3}\mkern#1#2_{#4}}}
\newcommand*{\perm}[1][-3mu]{\permcomb[#1]{P}}
\newcommand*{\comb}[1][-1mu]{\permcomb[#1]{C}}
\numberwithin{equation}{subsection}
\makeatletter
\@addtoreset{figure}{problem}
\makeatother
\let\StandardTheFigure\thefigure
\let\vec\mathbf
\renewcommand{\thefigure}{\theproblem}
\def\putbox#1#2#3{\makebox[0in][l]{\makebox[#1][l]{}\raisebox{\baselineskip}[0in][0in]{\raisebox{#2}[0in][0in]{#3}}}}
     \def\rightbox#1{\makebox[0in][r]{#1}}
     \def\centbox#1{\makebox[0in]{#1}}
     \def\topbox#1{\raisebox{-\baselineskip}[0in][0in]{#1}}
     \def\midbox#1{\raisebox{-0.5\baselineskip}[0in][0in]{#1}}
\vspace{3cm}
\title{Assignment 6}
\author{Sujal - AI20BTECH11020}
\maketitle
\newpage
\bigskip
\renewcommand{\thefigure}{\theenumi}
\renewcommand{\thetable}{\theenumi}
Download all latex codes from 

\begin{lstlisting}
https://github.com/https://github.com/sujal100/Probability_and_Random_variable/blob/main/exercise_6/exercise_6_main_tex.tex
\end{lstlisting}

\section{Problem [CSIR NET(JUNE-2017) MATHS-STATISTICS (Q-50)]}
$X_{1}, X_{2}, \cdots$ are independent identically distributed random variables
having common density $f$. Assume $f(x)=f(-x)$ for all $x \in \mathbb{R}$. Which of the following statements is correct?
\begin{enumerate}[label=\alph*)]
\item $\dfrac{1}{n}\left(X_{1}+\cdots+X_{n}\right) \rightarrow 0$ in probability
\item $\dfrac{1}{n}\left(X_{1}+\cdots+X_{n}\right) \rightarrow 0$ almost surely
\item $\pr{\dfrac{1}{\sqrt{n}}\left(X_{1}+\cdots+X_{n}\right)<0} \rightarrow \dfrac{1}{2}$
\item $\sum_{i=1}^{n} X_{i}$ has the same distribution as $\sum_{i=1}^{n}(-1)^{i} X_{i}$
\end{enumerate}

\section{Solution}
We using
\begin{enumerate}[label=\alph*)]
\item 
\begin{enumerate}[label=(\arabic*)]
\item {\em Convergence in probability : }Let $X_{1}, X_{2}, \ldots$ be an infinite sequence of random variables, and let $Y$ be another random variable. Then the sequence $\left\{X_{n}\right\}$ converges in probability to $Y$, if
\begin{align}
\forall \epsilon>0, \lim _{n \rightarrow \infty} \pr{\left|X_{n}-Y\right| \geq \epsilon}=0,
\end{align}
and we write 
\begin{align}
X_{n} \stackrel{P}{\rightarrow} Y.
\end{align}
\item {\em Convergence in almost surely : }Let $X_{1}, X_{2}, \ldots$ be an infinite sequence of random variables. We shall say that the sequence $\left\{X_{i}\right\}$ converges with probability 1 (or converges almost surely (a.s.)) to a random variable $Y$, if 
\begin{align}
\pr{\lim _{n \rightarrow \infty} X_{n}=Y}=1
\end{align}
and we write
\begin{align}
X_{n} \stackrel{a . s .}{\rightarrow} Y
\end{align}
\item {\em Strong law of large number(SLLN) : }Let $X_{1}, X_{2}, \ldots$ be an infinite sequence of random variables, If $\mathbf{E}\left[\left|X_{1}\right|\right]<\infty$. Then, as $n \rightarrow \infty$ , we have 
\begin{align}
\dfrac{S_{n}}{n} \stackrel{a.s.}{\rightarrow} \mathbf{E}\left[X_{1}\right]\implies \dfrac{S_{n}}{n} \stackrel{P}{\rightarrow} \mathbf{E}\left[X_{1}\right],
\label{eq} \\
\text{where} ,\,S_n = X_1 + \cdots + X_n
\end{align}  
\end{enumerate}
using SLLN, $(\mathrm{B})$ are incorrect option.
\item {\em Relation between in probability and almost surely : }Let $Z, Z_{1}, Z_{2}, \ldots$ be random variables. Suppose $Z_{n} \rightarrow Z$ with probability 1. Then ,we say 
\begin{align}
 Z_n \stackrel{a.s.}{\rightarrow} Z \implies Z_n \stackrel{P}{\rightarrow} Z.
\end{align}
\eqref{eq}, also in probability also hold this equation. Hence $(\mathrm{A})$ is incorrect option.  
\item 
{\em Central Limit Theorem : }Let $X_{1}, X_{2}, \ldots$ be i.i.d. with finite mean $\mu$ and finite variance $\sigma^{2}$. Let $Z \sim N(0,1) .$ Set $S_{n}=X_{1}+\cdots+X_{n}$, and
\begin{align}
Z_{n}=\frac{S_{n}-n \mu}{\sqrt{n \sigma^{2}}}
\end{align}
Then as $n \rightarrow \infty$, the sequence $\left\{Z_{n}\right\}$ converges in distribution to the $Z$, i.e., $Z_{n} \stackrel{D}{\rightarrow} Z$.\\
Consider,
\begin{align}
Y=\dfrac{X_{1}+ X_{2}+ \cdots + X_n}{\sqrt{n}}
\end{align}
So,
\begin{align}
E(Y)= E\left(\dfrac{X_{1}+ X_{2}+ \cdots+X_n}{\sqrt{n}}\right)= 0\\
V(Y)= V\left(\dfrac{X_{1}+ X_{2}+ \cdots+X_n}{\sqrt{n}}\right)= \frac{1}{n}2n=2
\end{align}
\begin{align}
Y \sim N[0,2]
\end{align}
we know that,
\begin{align}
f(x)=f(-x)\implies \text{Symmetry about Zero},
\end{align}
So,
\begin{align}
\pr{Y<0}=\frac{1}{2}
\end{align}
\begin{align}
\pr{\dfrac{1}{\sqrt{n}}\left(X_{1}+\cdots+X_{n}\right)<0}=\frac{1}{2}
\end{align}
Hence$,(\mathrm{C})$ is incorrect option.
\item {\em Characteristic function : }For a scalar random variable $X$ the characteristic function is defined as the expected value of $\mathrm{e}^{i t \times}$, where $i$ is the imaginary unit, and $t \in \mathbf{R}$ is the argument of the characteristic function:
\begin{align}
\left\{\begin{array}{l}
\varphi_{X}: \mathbb{R} \rightarrow \mathbb{C} \\
\varphi_{X}(t)=\mathrm{E}\left[e^{i t X}\right]=\int_{\mathbb{R}} e^{i t x} d F_{X}(x)\\=\int_{\mathbb{R}} e^{i t x} f_{X}(x) d x=\int_{0}^{1} e^{i t Q_{X}(p)} d p
\end{array}\right.
\end{align}
Here $F_X$ is the cumulative distribution function of $X$,Consider, $\phi_{x}(t)$ is characteristic function of $X_{i}, i=1, \ldots, n.$
\begin{align}
f(x)=f(-x)\implies \phi_{x}(t)=\phi_{-x}(t)
\end{align}
Therefore,
\begin{align}
\phi_{\sum_{i=1}^{n}X_i}(t) = \phi_{X_{1}+\ldots +X_{n}}(t) &=\phi_{X_{1}}(t)\cdots\phi_{X_{n}}(t)\\
&=\left[\phi_{x}(t)\right]^{n}
\end{align}
similarly,
\begin{align}
\phi_{\sum_{i=1}^{n}(-1)^{i}X_i}(t) &= \phi_{-X_{1}}+\phi_{X_{2}}+\ldots +\phi_{(-1)^{n}X_{n}}(t)\\
&=\phi_{-X_{1}}(t) \cdot \phi_{X_{2}}(t)\cdots\phi_{(-1)^{n}X_{n}}(t)\\
&=\left[\phi_{x}(t)\right]^{n}\\
\phi_{\sum_{i=1}^{n}X_i}(t) &= \phi_{\sum_{i=1}^{n}(-1)^{i}X_i}(t)
\end{align}
$\therefore \sum_{i=1}^{n} X_{i}$ has same distribution as $\sum_{i=1}^{n}(-1)^{i} X_{i}.$\\
Hence, only $(\mathrm{D})$ is correct option.
\end{enumerate}
\end{document}