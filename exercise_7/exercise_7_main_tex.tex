\documentclass[journal,12pt,twocolumn]{IEEEtran}

\usepackage{setspace}
\usepackage{gensymb}
\singlespacing
\usepackage[cmex10]{amsmath}
\usepackage{multirow}
\usepackage{amsthm}
\usepackage{mathrsfs}
\usepackage{txfonts}
\usepackage{stfloats}
\usepackage{bm}
\usepackage{cite}
\usepackage{cases}
\usepackage{subfig}

\usepackage{longtable}

\usepackage{enumitem}
\usepackage{mathtools}
\usepackage{steinmetz}
\usepackage{tikz}
\usepackage{circuitikz}
\usepackage{verbatim}
\usepackage{tfrupee}
\usepackage[breaklinks=true]{hyperref}
\usepackage{graphicx}
\usepackage{tkz-euclide}

\usetikzlibrary{calc,math}
\usepackage{listings}
    \usepackage{color}                                            %%
    \usepackage{array}                                            %%
    \usepackage{longtable}                                        %%
    \usepackage{calc}                                             %%
    \usepackage{multirow}                                         %%
    \usepackage{hhline}                                           %%
    \usepackage{ifthen}                                           %%
    \usepackage{lscape}     
\usepackage{multicol}
\usepackage{chngcntr}

\DeclareMathOperator*{\Res}{Res}

\renewcommand\thesection{\arabic{section}}
\renewcommand\thesubsection{\thesection.\arabic{subsection}}
\renewcommand\thesubsubsection{\thesubsection.\arabic{subsubsection}}

\renewcommand\thesectiondis{\arabic{section}}
\renewcommand\thesubsectiondis{\thesectiondis.\arabic{subsection}}
\renewcommand\thesubsubsectiondis{\thesubsectiondis.\arabic{subsubsection}}


\hyphenation{op-tical net-works semi-conduc-tor}
\def\inputGnumericTable{}                                 %%

\lstset{
%language=C,
frame=single, 
breaklines=true,
columns=fullflexible
}
\graphicspath{{./Figures/}}
\begin{document}


\newtheorem{theorem}{Theorem}[section]
\newtheorem{problem}{Problem}
\newtheorem{proposition}{Proposition}[section]
\newtheorem{lemma}{Lemma}[section]
\newtheorem{corollary}[theorem]{Corollary}
\newtheorem{example}{Example}[section]
\newtheorem{definition}[problem]{Definition}

\newcommand{\BEQA}{\begin{eqnarray}}
\newcommand{\EEQA}{\end{eqnarray}}
\newcommand{\define}{\stackrel{\triangle}{=}}
\bibliographystyle{IEEEtran}
\raggedbottom
\setlength{\parindent}{0pt}
\providecommand{\mbf}{\mathbf}
\providecommand{\pr}[1]{\ensuremath{\Pr\left(#1\right)}}
\providecommand{\qfunc}[1]{\ensuremath{Q\left(#1\right)}}
\providecommand{\sbrak}[1]{\ensuremath{{}\left[#1\right]}}
\providecommand{\lsbrak}[1]{\ensuremath{{}\left[#1\right.}}
\providecommand{\rsbrak}[1]{\ensuremath{{}\left.#1\right]}}
\providecommand{\brak}[1]{\ensuremath{\left(#1\right)}}
\providecommand{\lbrak}[1]{\ensuremath{\left(#1\right.}}
\providecommand{\rbrak}[1]{\ensuremath{\left.#1\right)}}
\providecommand{\cbrak}[1]{\ensuremath{\left\{#1\right\}}}
\providecommand{\lcbrak}[1]{\ensuremath{\left\{#1\right.}}
\providecommand{\rcbrak}[1]{\ensuremath{\left.#1\right\}}}
\theoremstyle{remark}
\newtheorem{rem}{Remark}
\newcommand{\sgn}{\mathop{\mathrm{sgn}}}
\providecommand{\abs}[1]{\left\vert#1\right\vert}
\providecommand{\res}[1]{\Res\displaylimits_{#1}} 
\providecommand{\norm}[1]{\left\lVert#1\right\rVert}
%\providecommand{\norm}[1]{\lVert#1\rVert}
\providecommand{\mtx}[1]{\mathbf{#1}}
\providecommand{\mean}[1]{E\left[ #1 \right]}
\providecommand{\fourier}{\overset{\mathcal{F}}{ \rightleftharpoons}}
%\providecommand{\hilbert}{\overset{\mathcal{H}}{ \rightleftharpoons}}
\providecommand{\system}{\overset{\mathcal{H}}{ \longleftrightarrow}}
	%\newcommand{\solution}[2]{\textbf{Solution:}{#1}}
\newcommand{\solution}{\noindent \textbf{Solution: }}
\newcommand{\cosec}{\,\text{cosec}\,}
\providecommand{\dec}[2]{\ensuremath{\overset{#1}{\underset{#2}{\gtrless}}}}
\newcommand{\myvec}[1]{\ensuremath{\begin{pmatrix}#1\end{pmatrix}}}
\newcommand{\mydet}[1]{\ensuremath{\begin{vmatrix}#1\end{vmatrix}}}
\newcommand*{\permcomb}[4][0mu]{{{}^{#3}\mkern#1#2_{#4}}}
\newcommand*{\perm}[1][-3mu]{\permcomb[#1]{P}}
\newcommand*{\comb}[1][-1mu]{\permcomb[#1]{C}}
\numberwithin{equation}{subsection}
\makeatletter
\@addtoreset{figure}{problem}
\makeatother
\let\StandardTheFigure\thefigure
\let\vec\mathbf
\renewcommand{\thefigure}{\theproblem}
\def\putbox#1#2#3{\makebox[0in][l]{\makebox[#1][l]{}\raisebox{\baselineskip}[0in][0in]{\raisebox{#2}[0in][0in]{#3}}}}
     \def\rightbox#1{\makebox[0in][r]{#1}}
     \def\centbox#1{\makebox[0in]{#1}}
     \def\topbox#1{\raisebox{-\baselineskip}[0in][0in]{#1}}
     \def\midbox#1{\raisebox{-0.5\baselineskip}[0in][0in]{#1}}
\vspace{3cm}
\title{Assignment 7}
\author{Sujal - AI20BTECH11020}
\maketitle
\newpage
\bigskip
\renewcommand{\thefigure}{\theenumi}
\renewcommand{\thetable}{\theenumi}
Download all latex codes from 

\begin{lstlisting}
https://github.com/https://github.com/sujal100/Probability_and_Random_variable/blob/main/exercise_6/exercise_6_main_tex.tex
\end{lstlisting}

\section{Problem [CSIR NET(JUNE-2017) MATHS-STATISTICS (Q-104)]}
Let $\left\{X_{n}, n \geq 1\right\}$ be i.i.d. uniform (-1,2) random variables. Which of the following statements are true?
\begin{enumerate}[label=\alph*)]
\item $\dfrac{1}{n} \sum_{i=1}^{n} X_{i} \rightarrow 0$ almost surely
\item $\left\{\dfrac{1}{2 n} \sum_{i=1}^{n} X_{2 i}-\dfrac{1}{2 n} \sum_{i=1}^{n} X_{2 i-1}\right\}\rightarrow 0$
almost surely
\item $\sup \left\{X_{1}, X_{2}, \ldots\right\}=2$ almost surely
\item $\inf \left\{X_{1}, X_{2}, \ldots\right\}=-1$ almost surely
\end{enumerate}

\section{Solution}
We using convergence in almost surely and Strong law of large number (SLLN)\\
Suppose $X_{n}, X$ are random variables on the same probability space. Then,\\
\begin{enumerate}
\item If $X_{n} \stackrel{\text { a.s. }}{\rightarrow} X$, then $X_{n} \stackrel{P}{\rightarrow} X$.
\item If $X_{n} \stackrel{P}{\rightarrow} X$ so that $\sum_{n} \mathbf{P}\left(\left|X_{n}-X\right|>\delta\right)<\infty$ for every $\delta>0$, then $X_{n} \stackrel{a_{\cdot B}}{\rightarrow} X$
\item (SLLN) Let $X_{n}$ be i.i.d with $\mathbf{E}\left[\left|X_{1}\right|\right]<\infty$. Then, as $n \rightarrow \infty$ , we have $\dfrac{S_{n}}{n} \stackrel{\text { a.s. }}{\rightarrow} \mathbf{E}\left[X_{1}\right]\implies \dfrac{S_{n}}{n} \stackrel{\text { P }}{\rightarrow} \mathbf{E}\left[X_{1}\right]  $, where $S_n = X_1 + \cdots + X_n.$
\item $X_i \stackrel{a.s.}{\rightarrow} X \implies g(X_i) \stackrel{a.s.}{\rightarrow} g(X)$
\end{enumerate}
\begin{enumerate}[label=\alph*)]
\item\begin{align}
\dfrac{1}{n}\left(X_{1}+\cdots+X_{n}\right) \rightarrow E(X)\in(-1,2)
\end{align}
as $n\rightarrow\infty,$ according to strong law of large numbers (SLLN).\\
So, option $(\mathrm{A})$ is incorrect.
\item \begin{align}
\left\{\dfrac{1}{2 n} \sum_{i=1}^{n} X_{2 i}-\dfrac{1}{2 n} \sum_{i=1}^{n} X_{2 i-1}\right\}&\stackrel{a.s.}{\rightarrow}\left\{\dfrac{nX}{2 n}-\dfrac{nX}{2 n}\right\}\\
&=0
\end{align}
option $(\mathrm{B})$ is correct.
\item using $X_i \stackrel{a.s.}{\rightarrow} X$ this , we also conclude that
\begin{align}
\sup \left\{X_{1}, X_{2}, \ldots\right\}&=2 \;almost\; surely\\
\inf \left\{X_{1}, X_{2}, \ldots\right\}&=-1 \;almost\; surely
\end{align}
Hence $(\mathrm{B}),(\mathrm{C})$ and $(\mathrm{D})$ are correct option.
\end{enumerate}
\end{document}